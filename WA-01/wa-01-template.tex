%%------------------------------------------------------
%% PLEASE DO NOT EDIT THIS 
%% SECTION UNLESS INSTRUCTED 
%% TO DO SO!

\documentclass{amsart}

\usepackage[utf8]{inputenc}
\usepackage[text={6in,9in},centering]{geometry} 
\usepackage{enumitem}
\usepackage[doublespacing]{setspace} % Makes the document double spaced.

\setlist[enumerate]{itemsep=3pt,topsep=3pt}
\setlist[enumerate,1]{label=(\alph*)}

\newcommand{\WAtitle}[2]{\noindent\textbf{\Large Writing Assignment \#{#1} \hfill due #2}\\}
\newcommand{\WAauthors}[1]{\noindent {#1}\\}
\newcommand{\defi}[1]{\textbf{\textit{#1}}}

\theoremstyle{definition}
\newtheorem*{prop*}{Proposition}
\newtheorem*{alignMe*}{Using Align Environment}



%% END OF DO NOT EDIT STUFF
%%------------------------------------------------------


%------------------------------------------------------
% Packages -- If you find new packages you like, use them here. 
%------------------------------------------------------
\usepackage{amssymb, amsmath, latexsym, amsfonts, amsthm, mathrsfs} % Standard packages that are nice to have.


%------------------------------------------------------
% New Macros to make life easier
%------------------------------------------------------
\DeclareMathOperator{\Row}{row}
\DeclareMathOperator{\Col}{col}
\DeclareMathOperator{\trace}{Tr}
\newcommand{\ds}{\displaystyle}


%------------------------------------------------------
% Main Document Body
%------------------------------------------------------
\begin{document}

\WAtitle{1}{January 30, 2019} %% Put the assignment number in the first { } and the due date in the second { }
\WAauthors{Rella Stone and Tipper Gibbons} %% Put the author here

\begin{prop*}[Screencast 1.2.4] (The statement you wish to prove should be written here; for example: ``For all square matrices $A$, if $A$ is symmetric, then $A^T$ is symmetric.'')
\end{prop*}

\begin{proof} 
Write the proof here as written in the video. Make sure to adhere to the writing guidelines posted on Blackboard. The goal of this assignment is to become familiar with both overleaf and \LaTeX. Feel free to run your solution by me before you turn it in. 
\end{proof}


\begin{alignMe*}[\S1.2, \#17a]
We wish to show that
\[
	\sum_{i = 1}^n (r_i+s_i)a_i = \sum_{i=1}^n r_ia_i + \sum_{i=1}^n s_ia_i.
\]
To see this, consider the following. 
\begin{align}
	\sum_{i = 1}^n (r_i+s_i)a_i &= (r_1+s_1)a_1 + (r_2+s_2)a_2 + \cdots + (r_n+s_n)a_n \label{def} \\
		&= \text{You finish the rest.} \label{finishme}\\
		&=
\end{align}	
Notice in \eqref{def} we used the definition of summation. In \eqref{finishme} we \ldots. 
\end{alignMe*}	

\begin{alignMe*}[\S1.2, \#17b]
	
\end{alignMe*}	

\begin{alignMe*}[\S1.2, \#18]
	
\end{alignMe*}	




\end{document}